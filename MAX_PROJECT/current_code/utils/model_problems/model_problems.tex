\documentclass{article}

\usepackage{amsmath}
\usepackage{amssymb}

\title{Test Matrices for Monte Carlo Synthetic Acceleration}

\begin{document}
\maketitle

\section{Introduction}

This document is intended to define a standard suite of linear
systems that can be used as test cases for Monte Carlo and
MCSA solvers.  The suite of matrices is intended to span a 
range of different matrix properties and problem difficulty.
This is intended to avoid potential pitfalls when performing
numerical experiments involving drawing conclusions based on
observations that may be specific to a particular linear system.

\section{Shifted 1-D Laplacian}

The most basic test matrix we propose is essentially a shifted
1-D discrete (negative) Laplacian operator.  In order to achieve a matrix
that is strongly diagonally dominant, we introduce a shift of
$2.0$ added to the main diagonal, resulting in the matrix
\begin{equation}
 \mathbf{A} = \begin{bmatrix} 4 & -1 \\ -1 & \ddots & \ddots \\ 
     & \ddots & \ddots & -1 \\  & & -1 & 4 \end{bmatrix} \:,
\end{equation}
which can be generated in MATLAB with the command
\begin{verbatim}
  A = 4*eye(N) - diag(ones(N-1),1) - diag(ones(N-1),-1);
\end{verbatim}
where $N$ is the size of the matrix.
The right hand side is taken to be
\begin{equation}
 \mathbf{b}_i = i, \quad i=0,\ldots,N-1 \:.
\end{equation}
A matrix of any size can be generated in this way and will have
similar numerical properties, but the sample matrices
provided here take $N=50$.
The spectral radius of $\mathbf{I - D^{-1}A}$ is equal to approximately
$0.499$.

\section{2-D Laplacian}

The second test problem is a finite difference discretization of the 2-D
Laplacian with homogeneous Dirichlet boundaries.  This matrix can be
generated in MATLAB by the command
\begin{verbatim}
  A = delsq(numgrid('S',M));
\end{verbatim}
where $M$ is the number of grid points in each direction.
As a right hand size, we take a sinusoid distribution in both the
$x$ and $y$ directions, i.e.
\begin{equation}
 \mathbf{b}(x_i,y_j) = \sin \left( \frac{\pi x_i}{M-1} \right) \;
                       \sin \left( \frac{\pi y_j}{M-1} \right), \quad i=0,\ldots,M-1 \:,
\end{equation}
where $x_i=\frac{i}{M}$ and $y_j=\frac{j}{M}$.  As with the previous example,
any size problem can be selected; however, for this matrix the problem becomes
more difficult as $M$ is increased.  We choose a value of $M=32$, for which
the size of the resulting matrix is 900 (the unknowns on Dirichlet boundaries
are eliminated from the matrix) and spectral radius of the iteration
matrix for this problem is approximately $0.9949$.

\section{IFISS Convection-Diffusion}

The next test problem is a finite element discretization of a 
2-D convection-diffusion problem using the MATLAB IFISS package.
Continuous bilinear elements on quadrilaterals (Q1 elements) are used.
A $32 \times 32$ mesh is used on the unit square with a viscosity of
$\nu=0.1$.  Streamline upwind Petrov--Galerkin (SUPG) stabilization
is used in the construction of the matrix.  Homogeneous Dirichlet
boundary conditions are applied on the top, bottom, and left sides of
the domain and a nonhomogeneous Dirichlet boundary with a constant value
of 1 is applied on the right side.  A constant forcing term with a value
of 1 is present throughout the domain.  The dimension of the resulting
matrix is 1089 and the spectral radius of $\mathbf{I - D^{-1}A}$
is 0.9835.

\section{JPWH Matrix}

The next test problem is the matrix ``JPWH\_991'' matrix available from
the Matrix Market.  This matrix
is described as a ``computer random simulation of a circuit physics model.''
We select the right hand side to be the constant vector of all ones.  The
matrix has dimension 991 and the
spectral radius of $\mathbf{I - D^{-1}A}$ is 0.9797.

\section{FS 680 1}

As another matrix market example we choose the matrix ``FS\_680\_1,''
which is described as a mixed kinetics diffusion problem from radiation chemistry.
The dimension of the problem is 680 and the spectral radius of the iteration matrix
is $0.9697$.  The primary reason for including this matrix is that approximately
half of the columns of the matrix are columns of the identity matrix.  This
results in the iteration matrix having a large number of zero columns and
therefore the transpose of the iteration matrix having a large number of 
zero rows.  This poses no problem for forward Monte Carlo calculations but
seems to result in very poor convergence behavior for the adjoint Monte Carlo
approach.  This is in contrast to typical behavior where the performance of
the adjoint Monte Carlo is typically significantly better than the forward
method.

\section{Marshak Wave}

The final test problem (so far) that we propose is from a Marshak wave problem
in thermal radiation diffusion.  The problem selected is a single time step
from a time dependent calculation.  An initial guess is provided which is the
solution at the previous time step.  The interesting feature for this problem
is that the solution only changes in the immediate vicinity of the wave front
and therefore the residual is very large within a small portion of the problem
domain and nearly zero everywhere else.  This presents a significant advantage
for the adjoint Monte Carlo method, which preferentially starts Monte Carlo
histories in locations where the residual is large.

\end{document}
