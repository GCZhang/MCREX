\section{Introduction}

For a long period the
pursuit of a reliable solver for linear and nonlinear algebraic systems has
been one of the most important issues in Numerical Analaysis. The
mathematical properties of the model (e.g. likely sparsity pattern of the
matrix) have urged scientists to look for efficient and pattern-driven
solvers, capable to cope with curse of dimensionality and high scalability
requests.

As scalability computing move towards exascale facilities (which stands for
machines computing up to $10^{18}$ flops), new numerical schemes suitable for
this
kind of architectures are strongly demanded. In fact the purpose is to combine
high
fidelity of the results with an optimized use of hardware resources at hand.

Recently new algorithms that combine statistical and numerical techniques have
been developed. Even though \textit{numerical efficiency} has been sacrificed
sometimes, the advantages in terms of \textit{robustness} of these methods make
it worth carrying on studies in this direction.
A class of methods that may be employed for this purpose is represented by
Monte Carlo linear solvers.

These algorithms are the main topic covered in the work presented here,
starting from what was developed so far by some of
the authors who gave significant contributions (see \cite{Hal1962},
\cite{Hal1994},
\cite{DA1998}, \cite{DVA2001}, \cite{AADBTW2005},\cite{ESW2013} and
\cite{EMSH2014}). \newline

As underlined in \cite{DA1998}, Monte Carlo methods may be split into two
classes: \textit{direct methods} (\cite{DA1998}, \cite{DVA2001}) and
\textit{iterative methods} (\cite{Hal1962},
\cite{Hal1994}, \cite{ESW2013}
and \cite{EMSH2014}). The first are characterized by a merely stochastic
scheme,
therefore the provided error with respect to the exact solution is made of
just a stochastic component. The second ones, instead, combine numerical and
statistical schemes generating two types of error: a
\textit{stochastic} one and a \textit{systematic} one. It does not
mean
that
it will be always simple to recognize them separately. However it is important
to
be
aware of this intrinsic structure, in order to target what is the part of the
scheme that requires a refinement (e.g. increase of numerical iterations
rather than random walks).\newline

The paper is organized as follows. In Section 2 we describe all the
viable stochastic settings for the definition of the Markovian process. In
Section 3 we reinterpret probabilistic facts in terms of spectral properties of
the linear system to be solved. Section 4 is dedicated to necessary conditions
and sufficient conditions for the convergence of the algorithms. In Section 5
different adaptive approaches are presented, in order to truncate the random
walks and to decide how many samplings to use. Section 6 regards the study
of matrices for which the convergence of the stochastic schemes is facilitated.
Then in Section 7 a hybrid scheme combining numerical and probabilistic
techniques is described: the \textit{Monte Carlo Synthetic Acceleration}.
Numerical results are presented in Section 8 and 9, while Section 10 is
dedicated to
conclusive remarks and goals for the future.

