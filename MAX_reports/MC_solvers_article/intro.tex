\section{Introduction}
\label{sec:intro}

For a long period the
pursuit of a reliable solver for linear and nonlinear algebraic systems has
been one of the most important issues in Numerical Analaysis. The
mathematical properties of the model (e.g. likely sparsity pattern of the
matrix) have urged scientists to look for efficient and pattern-driven
solvers, capable to cope with curse of dimensionality and high scalability
requests.

As scalability computing move towards exascale facilities (which stands for
machines computing up to $10^{18}$ flops), new numerical schemes suitable for
this
kind of architectures are strongly demanded. In fact the purpose is to combine
high
fidelity of the results with an optimized use of hardware resources at hand.

Recently new algorithms that combine statistical and numerical techniques have
been developed. Even though \textit{numerical efficiency} has been sacrificed
sometimes, the advantages in terms of \textit{robustness} of these methods make
it worth carrying on studies in this direction.
A class of methods that may be employed for this purpose is represented by
Monte Carlo linear solvers.

These algorithms are the main topic covered in the work presented here,
starting from what was developed so far by some of
the authors who gave significant contributions (see \cite{Hal1962},
\cite{Hal1994},
\cite{DA1998}, \cite{DVA2001}, \cite{AADBTW2005},\cite{ESW2013} and
\cite{EMSH2014}). \newline

As underlined in \cite{DA1998}, Monte Carlo methods may be split into two
classes: \textit{direct methods} (\cite{DA1998}, \cite{DVA2001}) and
\textit{iterative methods} (\cite{Hal1962},
\cite{Hal1994}, \cite{ESW2013}
and \cite{EMSH2014}). The first are characterized by a merely stochastic
scheme,
therefore the provided error with respect to the exact solution is made of
just a stochastic component. The iterative Monte Carlo methods utilize more
traditional iterative algorithms alongside the stochastic approach,
generating two types of error: a
\textit{stochastic} one and a \textit{systematic} one. It does not
mean
that
it will be always simple to recognize them separately. However it is important
to
be
aware of this intrinsic structure, in order to target what is the part of the
scheme that requires a refinement (e.g. increasing the number of iterations
rather than the number of random walks).

The paper is organized as follows.
In Section~\ref{sec:mcls} we provide an overview of existing Monte Carlo
linear solver algorithms.
In Section~\ref{sec:convergence} we will discuss the convergence behavior
of stochastic solvers, including a discussion of classes of matrices for
which convergence can be guaranteed.
Section~\ref{sec:results} provides some numerical results illustrating
pertinent properties of the various approaches and Section~\ref{sec:conclusion}
will provide some concluding remarks and areas for future investigation.

