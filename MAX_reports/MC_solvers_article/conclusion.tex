\section{Conclusions and Future Work}
\label{sec:conclusion}

The work presented in this paper has been accomplished to provide an overview
about MC linear solvers. In particular the goal was to combine the main
theoretical results found in literature with additional novel contributions on
both the theoretical and empirical sides. This made it possible to understand
the actual limitations of the presented algorithms.

Necessary and sufficient conditions to ensure convergence are too costly for
most of the applicative problems coming fro engineering and Computational
Physics. On the other hand, sufficient criteria are currently viable for a
restricted set of matrices (generalized diagonally dominant).

An empirical analysis has been carries out on parabolic problems and it has
been shown that a proper selection of the time discretization step enables the
employment of stochastic solvers.

We reserve for future works the analysis of the plausible connections between
$\rho(H)$, $\rho(\hat{H})$ and the number of numerical iterations necessary to
reach convergence. Moreover a study about the resilience of the problem will be
accomplished, resorting to fault injection techniques.
i
