%%---------------------------------------------------------------------------%%
%% abstract.tex
%%---------------------------------------------------------------------------%%

\hrule
\vskip10pt
\noindent
{\bf\large Abstract}.  

The next generation of computational science applications will require
numerical solvers that are capable of high performance on proposed exascale
platforms.  In order to meet this goal, solvers must be resilient to soft and
hard system failures, provide high concurrency on heterogeneous hardware
configurations, and retain numerical accuracy and efficiency.  In light of
these requirements, a natural avenue of inquiry would be to adapt the current
stable of numerically efficient solvers to this new high-performance computing
regime.  However, an alternative approach would be to investigate different
classes of algorithms that can address issues of resiliency, particularly
fault tolerance and hard processor failures, naturally.  In this proposal we
will investigate new stochastic methods for solving linear systems, otherwise
termed Monte Carlo Resilient, Exascale (MCREX) solvers.  The family of methods
that we have proposed builds on the sequential Monte Carlo work of Halton,
1962. While showing significant promise, this class of solvers has not made
inroads into the broader computational science community.  The methods that we
have initially developed use Monte Carlo to accelerate a fixed-point
iteration; therefore, we have called them Monte Carlo Synthetic Acceleration
(MCSA). Preliminary work using MCSA has demonstrated that they are at least as
efficient as Jacobi-preconditioned Conjugate Gradient (PCG) on sparse, SPD
systems.  These initial results demonstrate that, because MCSA does not
require symmetry or positive definiteness, very good efficiency could be
attained on non-symmetric systems, thus making MCSA an ideal solver in
non-linear Newton schemes.  Furthermore, Monte Carlo methods have the benefit
of addressing resiliency in a natural way; soft errors can be treated as high
variance samples and lost histories from processor failures can be easily
discarded without affecting the quality of the solution.
 
%%---------------------------------------------------------------------------%%
%% end of abstract.tex
%%---------------------------------------------------------------------------%%

