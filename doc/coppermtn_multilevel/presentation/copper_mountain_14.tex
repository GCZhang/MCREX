\documentclass{beamer}
\usetheme{Warsaw}
\usepackage{graphicx}
\usepackage{sidecap}
\usepackage{longtable}
\usepackage{listings}
\usepackage{color}
%% The amssymb package provides various useful mathematical symbols
\usepackage{amssymb}
%% The amsthm package provides extended theorem environments
\usepackage{amsthm} 
\usepackage{amsmath} 
\usepackage{tmadd,tmath}
\usepackage[mathcal]{euscript} 
\usepackage{color}
\usepackage{textcomp}
\usepackage{algorithm,algorithmic}
\definecolor{listinggray}{gray}{0.9}
\definecolor{lbcolor}{rgb}{0.9,0.9,0.9}
\lstset{
  backgroundcolor=\color{lbcolor},
  tabsize=4,
  rulecolor=,
  language=c++,
  basicstyle=\scriptsize,
  upquote=true,
  aboveskip={1.5\baselineskip},
  columns=fixed,
  showstringspaces=false,
  extendedchars=true,
  breaklines=true,
  prebreak =
  \raisebox{0ex}[0ex][0ex]{\ensuremath{\hookleftarrow}},
  frame=single,
  showtabs=false,
  showspaces=false,
  showstringspaces=false,
  identifierstyle=\ttfamily,
  keywordstyle=\color[rgb]{0,0,1},
  commentstyle=\color[rgb]{0.133,0.545,0.133},
  stringstyle=\color[rgb]{0.627,0.126,0.941},
}

%% colors
\setbeamercolor{boxheadcolor}{fg=white,bg=black}
\setbeamercolor{boxbodycolor}{fg=black,bg=white}

%% slide numbers

%%---------------------------------------------------------------------------%%
\author{Stuart Slattery}

\date{\today} 
\title[Multilevel Monte Carlo Solvers for Linear Systems \hspace{1mm}
  \insertframenumber/\inserttotalframenumber]{Multilevel Monte Carlo
  Solvers for Linear Systems}
\begin{document}
\maketitle

%%---------------------------------------------------------------------------%%
\begin{frame}{Hardware-Based Motivation}

  \begin{itemize}
  \item Modern hardware is moving in two directions (Kogge,2011):
    \begin{itemize}
    \item Lightweight machines
    \item Heterogeneous machines
    \item Both characterized by low power and high concurrency
    \end{itemize}
    \medskip \medskip
  \item Some issues:
    \begin{itemize}
    \item Higher potential for both soft and hard failures (DOE,2012)
    \item Memory restrictions are expected with a continued decrease
      in memory/FLOPS
    \end{itemize}
    \medskip \medskip
  \item Potential resolution from Monte Carlo:
    \begin{itemize}
    \item Soft failures buried within the tally variance
    \item Hard failures mitigated by replication
    \item Memory savings over conventional methods
    \end{itemize}
  \end{itemize}

\end{frame}

%%---------------------------------------------------------------------------%%
\begin{frame}{Monte Carlo Methods for Discrete Linear Systems}

  \begin{itemize}
  \item First proposed by J. Von Neumann and S.M. Ulam in the 1940's
    \medskip \medskip
  \item Earliest published reference in 1950
    \medskip \medskip
  \item General lack of published work
    \medskip \medskip
  \item Modern work by Evans and others has yielded new
    applications\let\thefootnote\relax\footnote{\tiny{Thomas Evans and
        Scott Mosher, "A Monte Carlo Synthetic Acceleration method for
        the non-linear, time-dependent diffusion equation", American
        Nuclear Society - International Conference on Mathematics,
        Computational Methods and Reactor Physics, 2009.}}
  \end{itemize}

\end{frame}

%%---------------------------------------------------------------------------%%
\begin{frame}{Monte Carlo Linear Solver Preliminaries}

  \begin{itemize}
  \item Split the linear operator
  \end{itemize}

  \[
  \ve{A}\ve{x} = \ve{b} \ \ \ \rightarrow \ \ \ \ve{x} = \ve{H} \ve{x}
  + \ve{b}
  \]

  \[
  \ve{H} = \ve{I} - \ve{A}
  \]

  \medskip
  \begin{itemize}
  \item Generate the \textit{Neumann series}
  \end{itemize}
  
  \[
  \ve{A}^{-1} = (\ve{I}-\ve{H})^{-1} = \sum_{k=0}^{\infty} \ve{H}^k
  \]

  \medskip
  \begin{itemize}
  \item Require $\rho(\ve{H}) < 1$ for convergence
  \end{itemize}

  \[
  \ve{A}^{-1}\ve{b} = \sum_{k=0}^{\infty} \ve{H}^k\ve{b} = \ve{x}
  \]

\end{frame}

%%---------------------------------------------------------------------------%%
\begin{frame}{Monte Carlo Linear Solver Preliminaries}

  \begin{itemize}
  \item Expand the Neumann series
  \end{itemize}

  \[
  x_i = \sum_{k=0}^{\infty}\sum_{i_1}^{N}\sum_{i_2}^{N}\ldots
  \sum_{i_k}^{N}h_{i,i_1}h_{i_1,i_2}\ldots h_{i_{k-1},i_k}b_{i_k}
  \]

  \begin{itemize}
  \item Define a sequence of state transitions
  \end{itemize}
  
  \[
  \nu = i \rightarrow i_1 \rightarrow \cdots \rightarrow i_{k-1}
  \rightarrow i_{k}
  \]

  \begin{itemize}
  \item Use the adjoint Neumann-Ulam
    decomposition\let\thefootnote\relax\footnote{The Hadamard product
      $\ve{A} = \ve{B} \circ \ve{C}$ is defined element-wise as
      $a_{ij} = b_{ij} c_{ij}$.}
  \end{itemize}

  \[
  \ve{H}^{T} = \ve{P} \circ \ve{W}
  \]

  \[
  p_{ij} = \frac{|h_{ji}|}{\sum_j |h_{ji}|},\ w_{ij} =
  \frac{h_{ji}}{p_{ij}}
  \]

\end{frame}

%%---------------------------------------------------------------------------%%
\begin{frame}{Evolution of a Solution}

  \begin{figure}[htpb!]
    \begin{center}
      \scalebox{1.0}{ \input{heat_eq_setup.pdftex_t} }
    \end{center}
    \caption{\textbf{Poisson Problem.}
      \textit{Distributed source of 1.0 in the domain.}}
  \end{figure}

\end{frame}

%%---------------------------------------------------------------------------%%
\begin{frame}{Evolution of a Solution}

  \begin{figure}[h!]
    \begin{center}
      \includegraphics<1>[width=4in]{adjoint_1.png}
      \includegraphics<2>[width=4in]{adjoint_10.png}
      \includegraphics<3>[width=4in]{adjoint_100.png}
      \includegraphics<4>[width=4in]{adjoint_1000.png}
      \includegraphics<5>[width=4in]{adjoint_10000.png}
      \includegraphics<6>[width=4in]{adjoint_100000.png}
      \includegraphics<7>[width=4in]{adjoint_1000000.png}
      \includegraphics<8>[width=4in]{adjoint_10000000.png}
    \end{center}
    \caption{
      \only<1>{\textbf{Adjoint solution to Poisson Equation.}
        \textit{\sn{1}{0} total histories, 0.286 seconds CPU time.} }
      \only<2>{\textbf{Adjoint solution to Poisson Equation.}
        \textit{\sn{1}{1} total histories, 0.278 seconds CPU time.} }
      \only<3>{\textbf{Adjoint solution to Poisson Equation.}
        \textit{\sn{1}{2} total histories, 0.275 seconds CPU time.} }
      \only<4>{\textbf{Adjoint solution to Poisson Equation.}
        \textit{\sn{1}{3} total histories, 0.291 seconds CPU time.} }
      \only<5>{\textbf{Adjoint solution to Poisson Equation.}
        \textit{\sn{1}{4} total histories, 0.428 seconds CPU time.} }
      \only<6>{\textbf{Adjoint solution to Poisson Equation.}
        \textit{\sn{1}{5} total histories, 1.76 seconds CPU time.} }
      \only<7>{\textbf{Adjoint solution to Poisson Equation.}
        \textit{\sn{1}{6} total histories, 15.1 seconds CPU time.} }
      \only<8>{\textbf{Adjoint solution to Poisson Equation.}
        \textit{\sn{1}{7} total histories, 149 seconds CPU time.} } 
    }
  \end{figure}

\end{frame}

%%---------------------------------------------------------------------------%%
\begin{frame}{Model Problem}

\end{frame}

%%---------------------------------------------------------------------------%%
\begin{frame}{Error Analysis}

\end{frame}

%%---------------------------------------------------------------------------%%
\begin{frame}{Multilevel Monte Carlo Methods}

\end{frame}

%%---------------------------------------------------------------------------%%
\begin{frame}{Multilevel Expectation}

\end{frame}

%%---------------------------------------------------------------------------%%
\begin{frame}{Multilevel Expectation}

\end{frame}

%%---------------------------------------------------------------------------%%
\begin{frame}[fragile]{Multilevel Monte Carlo Solver}

  \begin{algorithm}[H]
    \caption{Multilevel Monte Carlo Method}
    \label{alg:mlamc}
    \begin{algorithmic}[1]
      { \small
      \FOR{ l = 0...L }
      \STATE  $\ve{P}_l = P(\ve{A}_l)$
      \COMMENT{Build the prolongation and restriction operators for
        the $l^{th}$ level.}
      \STATE $\ve{R}_l = c \ve{P}_l^T$
      \STATE $\ve{r}_l = \ve{b}_l - \ve{A}_l \ve{x}_l^0$
      \COMMENT{Build the $l^{th}$ level residual.}
      \STATE $\ve{d}_l = \hat{\ve{A}}_l^{-1} \ve{r}_l$
      \COMMENT{Solve the $l^{th}$ level problem with adjoint Monte
        Carlo}
      \IF{ l != L }
      \STATE $\ve{d}_l = (\ve{I} - \ve{P}_l\ve{R}_l) \ve{d}_{l}$
      \COMMENT{Apply the multilevel tally}
      \STATE $\ve{A}_{l+1} = \ve{R}_l \ve{A}_l \ve{P}_l$
      \COMMENT{Construct the next level.}
      \STATE $\ve{x}_{l+1}^0 = \ve{R}_l \ve{x}_l^0$
      \STATE $\ve{b}_{l+1} = \ve{R}_l \ve{b}_l$
      \ENDIF
      \ENDFOR

      \FOR{ l = L...1 }
      \STATE $\ve{d}_{l-1} = ( \ve{I} + \ve{P}_{l} ) \ve{d}_{l}$
      \COMMENT{Collapse the tallies to the finest grid}
      \ENDFOR

      \STATE $\ve{x} = \ve{x}^0 + \ve{d}_0$
      }
    \end{algorithmic}
  \end{algorithm}

\end{frame}

%%---------------------------------------------------------------------------%%
\begin{frame}{Numerical Experiments}

\end{frame}

%%---------------------------------------------------------------------------%%
\begin{frame}{Geometric Multigrid Example}

\end{frame}

%%---------------------------------------------------------------------------%%
\begin{frame}{Geometric Multigrid Example}

\end{frame}

%%---------------------------------------------------------------------------%%
\begin{frame}{Algebraic Multigrid Example}

\end{frame}

%%---------------------------------------------------------------------------%%
\begin{frame}{Algebraic Multigrid Example}

\end{frame}

%%---------------------------------------------------------------------------%%
\begin{frame}{Summary}

\end{frame}

%%---------------------------------------------------------------------------%%

\end{document}
