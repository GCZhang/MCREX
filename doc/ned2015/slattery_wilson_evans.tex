\documentclass[preprint,11pt]{elsarticle}
\usepackage[top=1.0in,bottom=1.0in,left=1.5in,right=1.5in]{geometry}
\usepackage{verbatim}
\usepackage{amssymb}
\usepackage{graphicx}
\usepackage{algorithm}
\usepackage{algpseudocode}
\usepackage{longtable}
\usepackage{caption}
\usepackage{subcaption}
\usepackage{amsfonts}
\usepackage{amsmath}
\usepackage{amssymb,amsthm}
\usepackage[mathcal]{euscript}
\usepackage{tabularx}
\usepackage{tmadd,tmath}
\usepackage[usenames]{color}
\usepackage[
naturalnames = true, 
colorlinks = true, 
linkcolor = black,
anchorcolor = black,
citecolor = black,
menucolor = black,
urlcolor = blue
]{hyperref}

\usepackage{booktabs}
\biboptions{sort&compress}
\journal{Nuclear Engineering Design}

%%%%%%%%%%%%%%%%%%%%%%%%%%%%%%%%%%%%%%%%%%%%%%%%%%%%%%%%%%%%%%%%%%%%%
%
%   BEGIN DOCUMENT
%
%%%%%%%%%%%%%%%%%%%%%%%%%%%%%%%%%%%%%%%%%%%%%%%%%%%%%%%%%%%%%%%%%%%%%
\begin{document}

\begin{frontmatter}

  \title{A Spectral Analysis of the Domain Decomposed Monte Carlo Method for
    Linear Systems}

  \tnotetext[ornl-cr]{Notice: This manuscript has been authored by
    UT-Battelle, LLC, under contract DE-AC05-00OR22725 with the
    U.S. Department of Energy. The United States Government retains
    and the publisher, by accepting the article for publication,
    acknowledges that the United States Government retains a
    non-exclusive, paid-up, irrevocable, world-wide license to publish
    or reproduce the published form of this manuscript, or allow
    others to do so, for United States Government purposes.}

  \author[ornl]{Stuart R. Slattery\fnref{fn1}}
  \ead{slatterysr@ornl.gov}

  \fntext[fn1]{Computational Engineering and Energy Sciences Group,
    Computer Science and Mathematics Division}

  \author[ornl]{Thomas M. Evans\fnref{fn2}}
  \ead{evanstm@ornl.gov}

  \fntext[fn1]{Radiation Transport Group, Reactor and Nuclear Systems
    Division}

  \author[wisc]{Paul P.H. Wilson\fnref{fn3}}
  \ead{wilsonp@engr.wisc.edu}

  \fntext[fn3]{Engineering Physics Department, College of Engineering}

  \address[ornl]{Oak Ridge National Laboratory, 1 Bethel Valley Rd., Oak
    Ridge, TN 37831, U.S.A.}

  \address[wisc]{University of Wisconsin - Madison, 1500 Engineering Dr.,
    Madison, WI 53706, U.S.A.}

  %%---------------------------------------------------------------------------%%
  \begin{abstract}
    The domain decomposed behavior of the adjoint Neumann-Ulam Monte Carlo
    method for solving linear systems is analyzed using the spectral
    properties of the linear operator. Relationships for the average length of
    the adjoint random walks, a measure of convergence speed and serial
    performance, are made with respect to the eigenvalues of the linear
    operator. In addition, relationships for the effective optical thickness
    of a domain in the decomposition are presented based on the spectral
    analysis and diffusion theory. Using the effective optical thickness, the
    Wigner rational approximation and the mean chord approximation are applied
    to estimate the leakage fraction of random walks from a domain in the
    decomposition as a measure of parallel performance and potential
    communication costs. The one-speed, two-dimensional neutron diffusion
    equation is used as a model problem in numerical experiments to test the
    models for symmetric operators with spectral qualities similar to light
    water reactor problems. In general, the derived approximations show good
    agreement with random walk lengths and leakage fractions computed by the
    numerical experiments.
  \end{abstract}

  \begin{keyword}
    Monte Carlo \sep domain decomposition \sep MCSA \sep parallel \sep
    computing \sep linear solvers
  \end{keyword}

\end{frontmatter}

%%---------------------------------------------------------------------------%%
\section{Introduction}
\label{sec:intro}

An alternative approach to solving linear systems is to employ Monte Carlo
methods that sample a distribution with an expectation value equivalent to
that of the inverted linear operator. Such methods have been in existence for
decades with the earliest reference noted here an enjoyable manuscript
published in 1950 by Forsythe and Leibler \cite{forsythe_matrix_1950}. In
their outline, Forsythe and Leibler in fact credit the creation of this
technique to J. Von Neumann and S.M. Ulam some years earlier than its
publication. In 1952 Wasow provided a more formal explanation of Von Neumann
and Ulam's method \cite{wasow_note_1952} and Hammersley and Handscomb's 1964
monograph \cite{hammersley_monte_1964} and Spanier and Gelbard's 1969 book
\cite{spanier_monte_1969} present additional detail on a method adjoint to
that described by Neumann and Ulam. To improve their convergence, Halton
leveraged the Monte Carlo algorithm in an iterative refinement scheme where a
residual Monte Carlo solution serves as a correction to a fixed point
iteration \cite{halton_sequential_1962}. In recent years, the adjoint
Neumann-Ulam method has been leveraged in Monte Carlo Synthetic Acceleration
(MCSA) methods that, for certain classes of problems, are competitive with
conventional methods methods and are applicable to a wide variety of linear
systems \cite{evans_monte_2009,evans_monte_2014}. To apply these new MCSA
methods to engineering problems, our research aims to improve the underlying
Monte Carlo algorithms to handle large-scale problems in parallel computing
environments.

To date, parallel Neumann-Ulam methods have been limited to full domain
replication with parallelism exploited through individual random walks
\cite{alexandrov_efficient_1998}. In reactor physics Monte Carlo applications,
however, domain decomposition has been identified as a key principle in moving
forward in high performance computing to enable higher fidelity simulations
\cite{brunner_comparison_2006,brunner_efficient_2009,siegel_analysis_2012}. To
accomplish this, we recognize from the literature that random walks must be
transported from domain to domain as the simulation progresses and they
transition to states that are not in the local domain. Because we have chosen
a domain decomposition strategy in a parallel environment, this means that
communication of these random walks must occur between compute nodes owning
neighboring pieces of the global domain. We wish to characterize this
communication not only because communication is in general expensive, but also
because these nearest-neighbor communication sequences, specifically, have
poor algorithmic strong scaling due to the simultaneous reduction of
on-process work and increase in communication as processor counts grow and
local problem sizes shrink \cite{gropp_high-performance_2001}.

The purpose of this study is to provide a simple, analytic theory based on the
properties of the linear system that will allow for estimates of the domain
decomposed behavior of the adjoint Neumann-Ulam method. When solving problems
where the linear operator is symmetric, a host of analytic theories exist
based on the eigenvalue spectrum of the operator that characterize their
behavior in the context of deterministic linear solvers. Using past work,
these theories are adapted to the domain decomposed adjoint Neumann-Ulam
method using the one-speed, two-dimensional neutron diffusion equation. In
this paper we describe the adjoint Neumann-Ulam Monte Carlo method followed by
a presentation of the model problem. Using the linear system generated by the
discretization of the model problem, we use a spectral analysis to generate
analytic relations for the eigenvalues of the operator based on system
parameters. Using the eigenvalue spectra, we then build relationships to
characterize the transport of random walks in a decomposed domain and
the fraction of random walks that leak from a domain and will therefore have to
be communicated. Finally, we compare these analytic results to numerical
experiments conducted with the model problem and draw conclusions looking
towards future work.

%%---------------------------------------------------------------------------%%
\section{Adjoint Neumann-Ulam Method}
\label{sec:adjoint_nu}

We seek solutions of the general linear problem in the following form:
\begin{equation}
  \ve{A} \ve{x} = \ve{b}\:.
  \label{eq:linear_problem}
\end{equation}
Choosing the adjoint Monte Carlo method to invert the linear operator, we
define the linear system adjoint to Eq~(\ref{eq:linear_problem}):
\begin{equation}
  \ve{A}^T \ve{y} = \ve{d}\:,
  \label{eq:adjoint_linear_problem}
\end{equation}
where $\ve{y}$ and $\ve{d}$ are the yet to be determined adjoint solution and
source respectively and $\ve{A}^T$ is the adjoint operator. We can derive a
Monte Carlo estimator from the adjoint method that will also give the solution
vector, $\ve{x}$. We first rearrange Eq~(\ref{eq:adjoint_linear_problem}) as:
\begin{equation}
  \ve{y} = (\ve{I} - \ve{H}^T)^{-1} \ve{d}\:,
  \label{eq:adjoint_split_system_2}
\end{equation}
where $\ve{H} = \ve{I} - \ve{A}$ is the \textit{iteration matrix}.
This yields the adjoint Neumann series:
\begin{equation}
  \ve{y} = \sum_{k=0}^{\infty} (\ve{H}^T)^k\ve{d}\:.
  \label{eq:adjoint_neumann_series}
\end{equation}
For this series to converge, we require that the spectral radius of $\ve{H}$
must remain less than unity\footnote{$\ve{H}^T$ contains the same eigenvalues
  as $\ve{H}$ and therefore has the same spectral radius.}. We expand the
matrix-vector multiply element-wise in this summation to yield a sum of sums:
\begin{equation}
  y_i = \sum_{k=0}^{\infty}\sum_{i_1}^{N}\sum_{i_2}^{N}\ldots
  \sum_{i_k}^{N}h_{i_k,i_{k-1}}\ldots h_{i_2,i_1} h_{i_1,i} d_{i_k}\:.
  \label{eq:adjoint_neumann_solution}
\end{equation}
To find the solution to Eq~(\ref{eq:linear_problem}) we build an
estimator for the adjoint solution from this series expansion. The
adjoint estimator can be related to the solution by defining the
following inner product equivalence \cite{spanier_monte_1969}:
\begin{equation}
  \langle \ve{A}^T \ve{x}, \ve{y} \rangle = \langle \ve{x}, \ve{A}
  \ve{y} \rangle\:.
  \label{eq:adjoint_operator_product}
\end{equation}
From this definition it follows that:
\begin{equation}
  \langle \ve{x}, \ve{d} \rangle = \langle \ve{y}, \ve{b} \rangle\:.
  \label{eq:adjoint_vector_relation}
\end{equation}
We have 2 unknowns, $\ve{y}$ and $\ve{d}$, and therefore we require two
constraints to close the system. We use Eq~(\ref{eq:adjoint_vector_relation})
as the first constraint and as a second constraint we select:
\begin{equation}
  \ve{d} = \boldsymbol{\delta}_j\:,
  \label{eq:adjoint_second_constraint}
\end{equation}
where $\boldsymbol{\delta}_j$ is one of a set of vectors in which the $j^{th}$
component is the Kronecker delta function, $\delta_{i,j}$. If we apply
Eq~(\ref{eq:adjoint_second_constraint}) to our first constraint in
Eq~(\ref{eq:adjoint_vector_relation}) we get the following convenient
outcome:
\begin{equation}
  \langle \ve{y}, \ve{b} \rangle = \langle \ve{x},
  \boldsymbol{\delta}_j \rangle = x_j \:,
  \label{eq:inner_product_constraint}
\end{equation}
meaning that if we compute the inner product of the original source and the
adjoint solution using a delta function source, we recover one component of
the original solution.

In the Neumann-Ulam method, a Monte Carlo game is played to construct the
terms of the Neumann series in Eq~(\ref{eq:adjoint_neumann_solution}). We use
random walks of length $k$ to build the series of state transitions in the
summation:
\begin{equation}
  \nu = i_0 \rightarrow i_1 \rightarrow i_2 \rightarrow \cdots \rightarrow
  i_{k-1}\:,
  \label{eq:permutation}
\end{equation}
where $\nu$ is a particular random walk permutation. Starting a random walk
permutation requires the computation of an initial starting state, $i_0$, and
initial weight, $W_0$. The initial state, $i_0$, of the random walk is
determined by sampling the source vector $\ve{b}$ with probabilities:
\begin{equation}
  P_{(i_0=i)}(\nu) = \frac{|b_i|}{||\ve{b}||_1}\:,
  \label{eq:adjoint_source_probability}
\end{equation}
and with a corresponding starting weight of:
\begin{equation}
  W_0 = ||\ve{b}||_1 \frac{b_{i_0}}{|b_{i_0}|}\:,
  \label{eq:adjoint_starting_weight}
\end{equation}
which gives the additional useful relation:
\begin{equation}
  b_{i_0} = W_0 P_{(i_0=i)}\:.
  \label{eq:adjoint_source_definition}
\end{equation}
After a random walk is started, we construct the remaining terms in
Eq~(\ref{eq:adjoint_neumann_solution}) using a row-scaled transition probability
matrix and corresponding transition weights:
\begin{equation}
  p_{ij} = \frac{|h_{ji}|}{\sum_j |h_{ji}|}\:,\ w_{ij} =
  \frac{|h_{ji}|}{p_{ij}},
  \label{eq:adjoint_probability}
\end{equation}
where each row of the probability matrix, $\ve{P}_i$, gives a probability
distribution function for moving from row $i$ to a new row, $j$. As a random
walk is computed by sampling the probability matrix, every stochastic
transition from a state $i$ to a state $j$ has a corresponding weight,
$w_{ij}$, that is used to update the current weight of the random walk. At the
$m^{th}$ step of a random walk, its weight is:
\begin{equation}
  W_{m} = W_0 w_{i_0,i_1} w_{i_1,i_2} \cdots w_{i_{m-1},i_m}\:,
  \label{eq:direct_permutation_weight}
\end{equation}
with $W_0$ given by Eq~(\ref{eq:adjoint_starting_weight}). To stop a random
walk, we utilize a relative weight cutoff parameter:
\begin{equation}
  W_f = W_c W_0:\,
  \label{eq:relative_weight_cutoff}
\end{equation}
where $W_c$ is the \textit{weight cutoff}. The random walk will be terminated
after $m$ steps if $W_m < W_f$.

Using our result from Eq~(\ref{eq:inner_product_constraint}) the Monte Carlo
estimator for the solution in state $i$ for a particular random walk
permutation of length $k$ is:
\begin{equation}
  X_{\nu} = \sum_{m=0}^k W_{m} \delta_{i,i_m}\:,
  \label{eq:adjoint_permutation_contribution}
\end{equation}
where the Kronecker delta indicates that the tally contributes only in the
current state, $i_m$. Finally, the expectation value using all random walk
permutations is:
\begin{equation}
  E\{X\} = \sum_{\nu} P_{\nu} X_{\nu}\:
  \label{eq:adjoint_expectation_value}
\end{equation}
which if expanded with Eqs~(\ref{eq:adjoint_probability}) and
(\ref{eq:adjoint_source_definition}) directly recovers the exact solution:
\begin{equation}
  \begin{split}
    E\{X_j\} &=\sum_{k=0}^{\infty}\sum_{i_1}^{N}\sum_{i_2}^{N}\ldots
    \sum_{i_k}^{N} b_{i_0} p_{i_0,i_1}p_{i_1,i_2}\ldots
    p_{i_{k-1},i_k} w_{i_0,i_1}w_{i_1,i_2}\ldots
    w_{i_{k-1},i_k} \delta_{i_k,j} \\ &= x_{j}\:.
  \end{split}
  \label{eq:adjoint_expectation_expansion}
\end{equation}

%%---------------------------------------------------------------------------%%
\section{Model Problem}
\label{sec:model_problem}

For our numerical experiments, we choose the one-speed,
two-dimensional neutron diffusion equation as a model problem
\cite{duderstadt_nuclear_1976}:
\begin{equation}
  -\boldsymbol{\nabla} \cdot D \boldsymbol{\nabla} \phi + \Sigma_a
  \phi = S\:,
  \label{eq:diffusion_eq}
\end{equation}
where $\phi$ is the neutron flux, $\Sigma_a$ is the absorption cross
section, and $S$ is the source of neutrons. In addition, $D$ is the
diffusion coefficient defined as:
\begin{equation}
  D = \frac{1}{3 ( \Sigma_t - \bar{\mu}\Sigma_s )}\:,
  \label{eq:diffusion_coeff}
\end{equation}
where $\Sigma_s$ is the scattering cross section, $\Sigma_t = \Sigma_a
+ \Sigma_s$ is the total cross section, and $\bar{\mu}$ is the cosine
of the average scattering angle. For simplicity, we will take
$\bar{\mu} = 0$ for our analysis giving $D=(3 \Sigma_t)^{-1}$. In
addition, to further simplify we will assume a homogeneous domain such
that the cross sections remain constant throughout. Doing this permits
us to rewrite Eq~(\ref{eq:diffusion_eq}) as:
\begin{equation}
  -D \boldsymbol{\nabla}^2 \phi + \Sigma_a \phi = S\:.
  \label{eq:diffusion_eq_simple}
\end{equation}

We choose a finite difference scheme on a square Cartesian grid to discretize
the problem. We choose the 9-point stencil over a grid of size $h$ for the
Laplacian \cite{leveque_finite_2007}:
\begin{multline}
  \nabla^2_9\phi = \frac{1}{6h^2}[4 \phi_{i-1,j} + 4 \phi_{i+1,j}
    + 4 \phi_{i,j-1} + 4 \phi_{i,j+1} + \phi_{i-1,j-1}\\ +
    \phi_{i-1,j+1} + \phi_{i+1,j-1} + \phi_{i+1,j+1} - 20
    \phi_{i,j}]\:.
  \label{eq:nine_point_stencil}
\end{multline}
We then have the following linear system to solve:
\begin{multline}
  -\frac{1}{6h^2}[4 \phi_{i-1,j} + 4 \phi_{i+1,j} + 4
    \phi_{i,j-1} + 4 \phi_{i,j+1} + \phi_{i-1,j-1}\\ + \phi_{i-1,j+1}
    + \phi_{i+1,j-1} + \phi_{i+1,j+1} - 20 \phi_{i,j}] + \Sigma_a
  \phi_{i,j} = s_{i,j}\:,
  \label{eq:fd_system}
\end{multline}
and in operator form:
\begin{equation}
  \ve{D}\boldsymbol{\phi}=\ve{s}\:,
  \label{eq:operator_system}
\end{equation}
where $\ve{D}$ is the diffusion operator resulting from the finite difference
discretization, $\ve{s}$ is the source in vector form and $\boldsymbol{\phi}$
is the vector of unknown fluxes.

%%---------------------------------------------------------------------------%%
\subsection{Domain Decomposition of the Model Problem}
\label{subsec:model_problem_decomp}
To investigate domain-to-domain communication in a parallel Monte Carlo
algorithm, we must consider the required data with the correct parallel
decomposition. In the adjoint Neumann-Ulam method, the domain decomposition
consists of all states in the system that are local, and the probabilities and
weights for all state transitions possible in the local domain. Given an input
parallel decomposition for the linear operator, by definition in
Eq~(\ref{eq:adjoint_probability}) the probabilities and weights will have the
same parallel decomposition and underlying sparse graph as the input
matrix. In the case of the model problem, parallel decomposition of states and
subsequently the input matrix is determined by a spatial partitioning of the
grid.

Compared to the serial construction of the probabilities and weights, data for
all states that are required to be on-process must be collected in parallel so
that all possible random walks that leave the local domain may be correctly
communicated. As an example of walking through the sparse graph of the matrix
and parallel communication, Figure~\ref{fig:diffusion_graph} gives the
adjacency graph of a single state in the neutron diffusion matrix for the
model problem presented in Eq~(\ref{eq:fd_system}). The discretization stencil
of the diffusion problem dictates the structure of this graph and the states
to which a random walk may transition. Consider a random walk starting at node
$(i,j)$ in the mesh (the center node). If we play the Monte Carlo game to move
to a new state, then we may move to any of the nodes in this graph, including
the node at which we started with a probability proportional to the
coefficients of each node in Eq~(\ref{eq:fd_system}). If any grid point
adjacent to node $(i,j)$ is on another processor, the random walk must be
communicated to that processor to continue the walk.

\begin{figure}[ht!]
  \begin{center}
    \scalebox{1.25}{ \input{stencil_graph.pdftex_t} }
  \end{center}
  \caption{\textbf{Neutron diffusion equation stencil adjacency graph.}
    \textit{The structure of the graph comes from the discretization of the
      Laplacian operator describing the diffusion physics in
      Eq~(\ref{eq:nine_point_stencil}). A random walk at mesh point $(i,j)$
      may transition to all adjacent mesh points including itself. If a
      random walk transitions to a node that is not owned by the local
      processor, it must be communicated to that processor to continue the
      walk.}}
  \label{fig:diffusion_graph}
\end{figure}

%%---------------------------------------------------------------------------%%
\section{Spectral Analysis}
\label{sec:spectral_analysis}

The convergence of the Neumann series in
Eq~(\ref{eq:adjoint_neumann_series}) approximated by the Monte Carlo
solver is dependent on the eigenvalues of the iteration matrix. We
will compute these eigenvalues by assuming eigenfunctions of the form
\cite{leveque_finite_2007}:
\begin{equation}
  \Phi_{p,q}(x,y) = e^{2 \pi \imath p x} e^{2 \pi \imath q y}\:,
  \label{eq:eigenfunction_form}
\end{equation}
where different combinations of $p$ and $q$ represent the different
eigenmodes of the solution. As these are valid forms of the solution,
then the action of the linear operator on these eigenfunctions should
give the eigenvalues of the matrix as they exist on the unit circle in
the complex plane.

\subsection{Iteration Matrix Spectrum}
\label{subsec:iteration_spectrum}

We first compute the eigenvalues for the diffusion operator $\ve{D}$ by
applying the operator to the eigenfunctions and noting that $x=ih$ and $y=jh$
on a rectilinear Cartesian grid:
\begin{multline}
  \ve{D}\Phi_{p,q}(x,y) = \lambda_{p,q}(\ve{D})
  =\\ -\frac{D}{6h^2}\Big[4 e^{-2 \pi \imath p h} + 4 e^{2 \pi \imath
      p h} + 4 e^{-2 \pi \imath q h} + 4 e^{2 \pi \imath q h} + e^{-2
      \pi \imath p h} e^{-2 \pi \imath q h} \\ + e^{-2 \pi \imath p h}
    e^{2 \pi \imath q h} + e^{2 \pi \imath p h} e^{-2 \pi \imath q h}
    + e^{2 \pi \imath p h} e^{2 \pi \imath q h} - 20\Big] + \Sigma_a
  \:.
  \label{eq:deriv_diff_1}
\end{multline}
Using Euler's formula, we can collapse the exponentials to
trigonometric functions:
\begin{equation}
  \lambda_{p,q}(\ve{D}) = -\frac{D}{6h^2}[ 8 \cos(\pi p h) + 8
    \cos(\pi q h) + 4 \cos(\pi p h) \cos(\pi q h) - 20] + \Sigma_a\:.
  \label{eq:deriv_diff_2}
\end{equation}

As Eq~(\ref{eq:diffusion_eq}) is diagonally dominant, Jacobi preconditioning
is sufficient to reduce the spectral radius of the iteration matrix below
unity and therefore ensure convergence of the Neumann series. The
preconditioner in this case is $\ve{M} = diag(\ve{D})$ such that we are
solving the following linear system:
\begin{equation}
  \ve{M}^{-1} \ve{D} \boldsymbol{\phi} = \ve{M}^{-1} \ve{s}\:.
  \label{eq:precond_diffsion}
\end{equation}
The operator $\ve{M}^{-1} \ve{D}$ is merely the original diffusion operator
with each row scaled by the diagonal component. As we have defined a
homogeneous domain, the scaling factor, $\alpha$, is the same for all rows in
the operator and defined as the $\phi_{i,j}$ coefficient from
Eq~(\ref{eq:fd_system}):
\begin{equation}
  \alpha = \Bigg[\frac{10 D}{3 h^2} + \Sigma_a\Bigg]^{-1}\:.
  \label{eq:jacobi_scaling}
\end{equation}
Using this coefficient, we have the following spectrum of preconditioned
eigenvalues:
\begin{equation}
  \lambda_{p,q}(\ve{M}^{-1} \ve{D}) = \alpha \lambda_{p,q}(\ve{D})\:.
  \label{eq:preconditioned_eigenvalues}
\end{equation}

The spectral radius of the iteration matrix is obtained by seeking its largest
eigenvalue. As with the diffusion operator, we can use the same analysis
techniques to find the eigenvalues for the iteration matrix. We use a few
simplifications by noting that if the Jacobi preconditioned iteration matrix
is $\ve{H} = \ve{I} - \ve{M}^{-1}\ve{D}$, then we expect all terms on the
diagonal of the iteration matrix to be zero such that we have the following
stencil:
\begin{equation}
  \ve{H}\boldsymbol{\phi} = \frac{\alpha D}{6h^2}[4 \phi_{i-1,j}
    + 4 \phi_{i+1,j} + 4 \phi_{i,j-1} + 4 \phi_{i,j+1} +
    \phi_{i-1,j-1} + \phi_{i-1,j+1} + \phi_{i+1,j-1} +
    \phi_{i+1,j+1}]\:.
  \label{eq:iteration_stencil}
\end{equation}
Inserting the eigenfunctions defined by Eq~(\ref{eq:eigenfunction_form}) we
get:
\begin{multline}
  \lambda_{p,q}(\ve{H}) = \frac{\alpha D}{6h^2}\Big[4 e^{-2 \pi \imath p
      h} + 4 e^{2 \pi \imath p h} + 4 e^{-2 \pi \imath q h} + 4 e^{2
      \pi \imath q h} + e^{-2 \pi \imath p h} e^{-2 \pi \imath q h}
    \\ + e^{-2 \pi \imath p h} e^{2 \pi \imath q q} + e^{2 \pi \imath
      p h} e^{-2 \pi \imath q h} + e^{2 \pi \imath p h} e^{2 \pi
      \imath q h}\Big]\:,
  \label{eq:iteration_deriv}
\end{multline}
which simplifies to:
\begin{equation}
  \lambda_{p,q}(\ve{H}) = \frac{\alpha D}{6h^2}[ 8 \cos(\pi p h) + 8
    \cos(\pi q h) + 4 \cos(\pi p h) \cos(\pi q h)]\:,
  \label{eq:iteration_spectrum}
\end{equation}
giving the eigenvalue spectrum of the Jacobi preconditioned iteration
matrix. We find that the maximum eigenvalue exists when $p=q=0$ giving the
following spectral radius of the Jacobi preconditioned iteration matrix:
\begin{equation}
  \rho(\ve{H}) = \frac{10 \alpha D}{3 h^2}\:.
  \label{eq:iteration_radius}
\end{equation}

\subsection{Neumann Series Convergence}
\label{subsec:neumann_convergence}

The adjoint Monte Carlo method is effectively an approximation to a
stationary method. Stationary methods for linear systems arise from
splitting the operator in Eq~(\ref{eq:linear_problem}) and iterating:
\begin{equation}
  \ve{x}^{k+1} = \ve{H}\ve{x}^k + \ve{c}\:,
  \label{eq:linear_iterative_method}
\end{equation}
with $k \in \mathbb{Z}^+$ defined as the \textit{iteration
  index}. Defining $\ve{e}^k = \ve{u}^k - \ve{u}$ as the solution
error at the $k^{th}$ iterate, the error after $k$ iterations is:
\begin{equation}
  \ve{e}^{k} = \ve{H}^k\ve{e}^0\:. 
  \label{eq:linear_k_iter_error}
\end{equation}
By assuming $\ve{H}$ is diagonalizable \cite{leveque_finite_2007}, we have:
\begin{equation}
  ||\ve{e}^{k}||_2 \leq \rho(\ve{H})^k ||\ve{e}^0||_2\:.
  \label{eq:linear_k_iter_norm3}
\end{equation}

In the adjoint Neumann-Ulam method, $k$ iterations, equivalent to $k$
applications of the iteration matrix, are approximated by a random walk of
average length $k$ to yield the summation in
Eq~(\ref{eq:adjoint_neumann_solution})
\cite{dimov_new_1998,danilov_asymptotic_2000}. This random walk length, or the
number of transitions before the termination of a random walk is therefore
approximately the number of stationary iterations required to converge to the
specified tolerance. In the case of the adjoint Neumann-Ulam method, no such
tolerance exists. However, we have specified a weight cutoff, $W_c$, that
determines when low-weight random walks will be terminated as their
contributions are deemed minute. After $k$ iterations, a stationary method is
terminated as the error has reached some fraction, $\epsilon$, of the initial
error:
\begin{equation}
  ||\ve{e}^{k}||_2 = \epsilon ||\ve{e}^0||_2\:.
  \label{eq:linear_k_iter_norm4}
\end{equation}
Per Eq~(\ref{eq:linear_k_iter_norm3}), we see that this fraction is equivalent
to $\epsilon = \rho(\ve{H})^k$. In the adjoint method, if we take this
fraction to be the weight cutoff, a measure of how accurately the
contributions of a particular random walk to the solution are tallied, we have
the following relationship for $k$:
\begin{equation}
  k = \frac{ \log(W_c) }{ \log( \rho(\ve{H}) ) }\:.
  \label{eq:analytic_k}
\end{equation}
This gives us a means to estimate the length of the random walks that will be
generated from a particular linear operator based on the eigenvalues of its
iteration matrix (independent of the linear operator splitting chosen) and
based on the weight cutoff parameter used in the Neumann-Ulam method.

\subsection{Domain Leakage Approximations}
\label{subsec:domain_leak_approx}

In a domain decomposed situation, not all random walks will remain within the
domain they started in and must instead be communicated. This communication,
expected to be expensive, was analyzed by Siegel and colleagues for idealized,
load balanced situations for full nuclear reactor core Monte Carlo simulations
\cite{siegel_analysis_2012}.  To quantify the number of particles that leak
out of the local domain they define a leakage fraction, $\Lambda$, as:
\begin{equation}
  \Lambda = \frac{\text{average \# of particles leaving local domain}}
          {\text{total \# of particles starting in local domain}}\:.
          \label{eq:leakage_fraction}
\end{equation}
For their studies, it was assumed that the value of $\Lambda$ was dependent on
the total cross section of the system via the Wigner rational
approximation. Outlined more thoroughly by Hwang's chapter in
\cite{azmy_nuclear_2010}, we will use both the Wigner rational approximation
and the mean chord approximation as a means to estimate the leakage fraction.

In the case of parallel linear systems, we can use diffusion theory to
estimate the optical thickness of a domain in the decomposition and the
corresponding leakage fraction in terms of properties of the linear operator
and the discretization. We first calculate the mean distance a random walk
will move in the grid by computing the mean squared distance of its movement
along the chord of length $l$ defined along a cardinal direction of the
domain. After a single transition a random walk will have moved a mean squared
distance of:
\begin{equation}
  \langle \bar{r_1^2} \rangle = (n_s h)^2\:,
  \label{eq:step_1_length}
\end{equation}
where $h$ is the size of the discrete grid elements along the chord and $n_s$
is the number of grid elements a random walk will move on average every
transition in the direction of the chord. For our diffusion model problem,
$n_s$ would equate to the expected number of states in the $i$ (or $j$ as the
problem is symmetric) direction that a random walk will move in a single
transition and is dependent on the stencil given in
Figure~\ref{fig:diffusion_graph}. After $k$ transitions, the random walk will
have moved a mean squared distance of:
\begin{equation}
  \langle \bar{r_k^2} \rangle = k (n_s h)^2\:.
  \label{eq:step_k_length}
\end{equation}
If our chord is of length $l$ and there are $n_i$ grid elements (or
states to which a random walk may transition) along that chord, then $h =
l / n_i$ giving:
\begin{equation}
  \langle \bar{r_k^2} \rangle = k \Bigg(\frac{n_s l}{n_i}\Bigg)^2\:.
  \label{eq:step_k_length_sub}
\end{equation}
From diffusion theory, we expect the average number of interactions
along the chord to be:
\begin{equation}
  \tau = \frac{l}{2 d \sqrt{\langle \bar{r_k^2} \rangle}}\:,
  \label{eq:optical_thickness_1}
\end{equation}
where $d$ is the dimensionality of the problem and $\sqrt{\langle
  \bar{r_k^2} \rangle}$ is effectively the mean free path of the Monte
Carlo random walk in the domain. We can readily interpret $\tau$ to be the
\textit{effective optical thickness} of a domain of length
$l$. Inserting Eq~(\ref{eq:step_k_length_sub}) we get:
\begin{equation}
  \tau = \frac{n_i}{2 d n_s \sqrt{k}}\:,
  \label{eq:optical_thickness_2}
\end{equation}
which if expanded with Eq~(\ref{eq:analytic_k}) gives us the final
relation for the effective optical thickness:
\begin{equation}
  \tau = \frac{n_i}{2 d n_s}
  \sqrt{\frac{\log(\rho(\ve{H}))}{\log(W_c)}}\:.
  \label{eq:optical_thickness_3}
\end{equation}

For optically thin domains, we expect that most random walks will be
communicated, while optically thick domains will leak the fraction of random
walks that did not interact within. Using the optical thickness defined in
Eq~(\ref{eq:optical_thickness_3}), we can complete the leakage approximations
by defining the bounds of $\tau \rightarrow 0, \Lambda \rightarrow 1$ and
$\tau \rightarrow \infty, \Lambda \rightarrow \tau^{-1}$.  With these bounds
we define the leakage fraction out of a domain using the Wigner rational
approximation:
\begin{equation}
  \Lambda = \frac{1}{1+\tau}\:,
  \label{eq:wigner_domain_leakage}
\end{equation}
and using the mean-chord approximation:
\begin{equation}
  \Lambda = \frac{1-e^{-\tau}}{\tau}\:.
  \label{eq:mean_chord_domain_leakage}
\end{equation}
Here, the leakage fraction is explicitly bound to the eigenvalues of
the iteration matrix, the size of the domain, the content of the
discretization stencil, and the weight cutoff selected to terminate
low weight random walks.

%%---------------------------------------------------------------------------%%
\section{Numerical Experiments}
\label{sec:numerical_experiments}

To test the relationships developed by the spectral analysis, we form two
simple numerical experiments using the diffusion model problem: one to measure
the length of the random walks as a function of the iteration matrix
eigenvalues, and one to measure the domain leakage fraction as a function of
the iteration matrix eigenvalues and the discretization properties. Before
doing this, we verify our computation of the spectral radius of the iteration
matrix by numerically computing the largest eigenvalue of the diffusion
operator using an iterative eigenvalue solver. For this verification, a $100
\times 100$ square grid with $h=0.01$, $h=0.1$, and $h=1.0$ was used and the
absorption cross varied from 0 to 100 while the scattering cross section was
fixed at unity. Figure~\ref{fig:measured_spec_rad} gives the measured spectral
radius of the iteration matrix and the computed spectral radius for the
preconditioned diffusion operator using Eq~(\ref{eq:iteration_radius}) as
function of the absorption to scattering ratio $(\Sigma_a /
\Sigma_s)$. Excellent agreement was observed between the analytic and
numerical results with all data points computed within the tolerance of the
iterative eigenvalue solver.
\begin{figure}[ht!]
  \begin{center}
    \includegraphics[width=0.8\textwidth]{spec_rad_results.eps}
  \end{center}
  \caption{Measured and analytic preconditioned diffusion operator
    spectral radius as a function of the absorption cross section to
    scattering cross section ratio. Values of $h=0.01$, $h=0.1$, and
    $h=1.0$ were used. The colored data was computed numerically by an
    eigensolver while the black dashed data was generated by
    Eq~(\ref{eq:iteration_radius}).}
  \label{fig:measured_spec_rad}
\end{figure}

\subsection{Random Walk Length}
\label{subsec:walk_length}

We next create an experiment to measure the length of the random walks
generated by the adjoint Neumann-Ulam solver. To do this, we again use a $100
\times 100$ square grid with $h=0.1$ and the absorption cross varied from 0 to
100 while the scattering cross section was fixed at unity. Three weight cutoff
values of \sn{1}{-2}, \sn{1}{-4}, and \sn{1}{-8} were used with 10,000 random
walks generated by a point source of strength 1 in the center of the
domain. For each of the random walks, the number of transitions made was
tallied to provide an effective value of $k$ for each random walk. This value
was then averaged over all random walks to get a measured value of $k$ for the
particular operator. Figure~\ref{fig:measured_length} presents
these measurements as well as the analytic result computed by
Eq~(\ref{eq:analytic_k}) as a function of the iteration matrix spectral
radius, $\rho(\ve{H})$. Figure~\ref{fig:measured_length_error} gives
the relative error between the predicted and observed results. We note good
qualitative agreement between the measured and analytic results. However, we
observe a larger relative error for both long and short random walks.
\begin{figure}[ht!]
  \begin{center}
    \includegraphics[width=0.8\textwidth]{random_walk_length.eps}
  \end{center}
  \caption{Measured and analytic random walk length as a function of the
    iteration matrix spectral radius. The weight cutoff was varied with
    \sn{1}{-2}, \sn{1}{-4}, and \sn{1}{-8}. The colored data was computed
    numerically by an adjoint Neumann-Ulam implementation while the black
    dashed data was generated by Eq~(\ref{eq:analytic_k}).}
  \label{fig:measured_length}
\end{figure}
\begin{figure}[ht!]
  \begin{center}
    \includegraphics[width=0.8\textwidth]{random_walk_error.eps}
  \end{center}
  \caption{Relative error between computed random walk length analytic length
    computed by Eq~(\ref{eq:analytic_k}) as a function of the iteration matrix
    spectral radius. The weight cutoff was varied with \sn{1}{-2}, \sn{1}{-4},
    and \sn{1}{-8}.}
  \label{fig:measured_length_error}
\end{figure}

\subsection{Domain Leakage}
\label{subsec:domain_leakage}

Finally, we measure the leakage of random walks from a domain in a Monte Carlo
calculation and assess the quality of our analytic relation for the optical
thickness of the domain and the associated leakage approximations. For this
experiment, a square grid with $h=0.1$ was decomposed into 9 square domains, 3
in each cardinal direction with measurements occurring in the central domain
without boundary grid points. For cross sections, the absorption cross section
was varied from 1 to 100 while the scattering cross section was set to zero to
create a purely absorbing environment with weight cutoff of \sn{1}{-4}. The
optical thickness of these domains will vary as a function of the absorption
cross section if the other parameters are fixed.

To compute the optical thickness, along with the spectral radius as given by
Eq~(\ref{eq:iteration_radius}), we also need the parameters $n_i$ and $n_s$
which respectively describe the typical domain length and the average number
of states moved along that typical length per random walk transition. For our
two-dimensional Cartesian grid, the domains are varied in size with $50 \times
50$, $100 \times 100$, and $200 \times 200$ cells giving $n_i=50$, $n_i=100$,
and $n_i=200$ grid points or states along the typical length of the domain
respectively.

To compute $n_s$, again consider the graph of the diffusion matrix stencil
given in Figure~\ref{fig:diffusion_graph}. We see that all random walk
transitions will only move a single state in either the $i$ or $j$ directions
due to the symmetry of the problem. Furthermore, if we choose a chord in the
$i$ direction, not all states to which we will transition move the random walk
in that direction. Instead, $n_s$ will be representative of the average
movement in the $i$ direction based on probabilities defined by the definition
of the iteration matrix in Eq~(\ref{eq:iteration_stencil}) and the definition
of the adjoint probability matrix in Eq~(\ref{eq:adjoint_probability}). For a
particular transition starting at state $(i,j)$, 6 of the 8 possible new
states in the stencil move the random walk in $i$ direction with relative
coefficients of 4 for moving in the $(\pm i,0)$ direction and of 1 for moving
in the $(\pm i,\pm j)$. These coefficients dictate the frequency those states
are visited relative to the others. For those 6 states we can visit along the
typical length, their sum is 12 out of the total 20 for the coefficients for
all possible states with their ratio giving $n_s = \frac{3}{5}$.

To compute the leakage fraction numerically, \sn{3}{5} random walks were
sampled from a uniform source of strength unity over the global domain. At the
start of a stage of random walks, the number of random walks starting in the
center domain was computed and as the stage progressed, the number of random
walks that exited that domain was tallied with the ratio of the two numbers
providing a numerical measure of the leakage
fraction. Figure~\ref{fig:measured_leakage} gives the domain leakage
measurements for the domain in the center of the global grid as well as the
analytic result computed by Eqs~(\ref{eq:wigner_domain_leakage}) and
(\ref{eq:mean_chord_domain_leakage}) as a function of the iteration matrix
spectral radius.
\begin{figure}[ht!]
  \begin{center}
    \includegraphics[width=0.8\textwidth]{leakage_fraction.eps}
  \end{center}
  \caption{Measured and analytic domain leakage as a function of the iteration
    matrix spectral radius. To test the behavior with respect to domain size,
    $n_i=50$, $n_i=100$,and $n_i=200$ were used. The colored data was computed
    numerically by a domain-decomposed adjoint Neumann-Ulam implementation,
    the black dashed data was generated by
    Eq~(\ref{eq:mean_chord_domain_leakage}) using the mean-chord
    approximation, and the dashed-dotted black data was generated by
    Eq~(\ref{eq:wigner_domain_leakage}) using the Wigner rational
    approximation.}
  \label{fig:measured_leakage}
\end{figure}
Again, we note good qualitative agreement between the measured and
analytic quantities but we begin to see the limits of the leakage
approximations. To compare the quality of the two approximations, the
absolute error between the computed leakage fraction and that
generated by the Wigner rational and mean chord approximations is
plotted in Figure~\ref{fig:leakage_error} for all domain sizes
tested. 
\begin{figure}[ht!]
  \begin{center}
    \includegraphics[width=0.8\textwidth]{leakage_error.eps}
  \end{center}
  \caption{Measured and analytic domain leakage absolute error as a
    function of the iteration matrix spectral radius.  To test the
    behavior with respect to domain size, $n_i=50$,
    $n_i=100$, and $n_i=200$ were used. The dashed
    lines represent the error using the Wigner rational
    approximation while the solid lines represent the error using
    the mean-chord approximation.}
  \label{fig:leakage_error}
\end{figure}
From these error results, the mean chord approximation is shown to have a
lower error for ill-conditioned systems as compared to the Wigner
approximation while the Wigner approximation produces less error for more
well-conditioned systems. We also note that for optically thick domains, the
error is likely corresponded to that observed in
Figure~\ref{fig:measured_length} for the $k$ parameter while the large
relative error in $k$ for optically thin domains does not affect the
approximation significantly. In general, the mean chord approximation is a
better choice to estimate the leakage fraction in a domain from the adjoint
Neumann-Ulam method and except for a single data point with $n_i=50$, the mean
chord approximation yielded leakage fractions within 0.05 of the measured
results. As the domain becomes more optically thick (with both increasing
$n_i$ and decreasing $\rho(\ve{H})$), the approximations are more accurate.

%%---------------------------------------------------------------------------%%
\section{Conclusion}

We have presented an analytic analysis of the domain decomposed
behavior of the adjoint Neumann Ulam method for linear systems. Good
agreement was observed for the derived analytic relationships for
random walk length and leakage fraction when compared to numerical
results generated by a domain decomposed implementation of the adjoint
method.

In future work, these relationships will serve as guidelines for
selecting an appropriate parallel algorithm strategy and provide a
basis for performance models for future parallel implementations of
the adjoint Neumann-Ulam method. To be applicable to a broader range
of problems, these models could potentially be extended to
non-symmetric systems. In addition, these relations will be used to
analyze parallel implementations of Monte Carlo synthetic acceleration
methods that leverage the adjoint method and serve as an initial
grounds for assessing their feasibility for large-scale problems.

%%---------------------------------------------------------------------------%%
\section{Acknowledgments}
This work was performed under appointment to the Nuclear Regulatory
Commission Fellowship program at the University of Wisconsin - Madison
Engineering Physics Department.

This material is based upon work supported by the U.S. Department of
Energy, Office of Science, Advanced Scientific Computing Research
program.

%%---------------------------------------------------------------------------%%
\bibliographystyle{ieeetr} \bibliography{references}

\end{document}


